\documentclass[a4paper]{article}

%% Language and font encodings
\usepackage[english]{babel}
\usepackage{amsthm}
\usepackage[utf8x]{inputenc}
\usepackage[T1]{fontenc}
\usepackage{listings}
\usepackage{algpseudocode, algorithm}
\usepackage{mathtools}
\usepackage{enumerate}
\usepackage{tabu}

%% Sets page size and margins
\usepackage[a4paper,top=3cm,bottom=2cm,left=3cm,right=3cm,marginparwidth=1.75cm]{geometry}

%% Useful packages
\usepackage{amsmath}
\usepackage{graphicx}
\usepackage[colorinlistoftodos]{todonotes}
\usepackage[colorlinks=true, allcolors=blue]{hyperref}

\newenvironment{solution}{\begin{proof}[\textnormal{\textbf{Solution}}]}{\end{proof}}
\newenvironment{exercise}[1]{\begin{proof}[\textnormal{\textbf{Exercise #1:}}]\phantom{\qedhere}}{\end{proof}}

\theoremstyle{definition}
\newtheorem{definition}{Definition}
\newtheorem{theorem}{Theorem}
\newtheorem{lemma}{Lemma}

\begin{document}
\begin{titlepage}\pagenumbering{gobble}
	\centering
	{\scshape\LARGE Factorial Problem\par}
	\vspace{1cm}
	{\Large\itshape Khalid Hourani\par}
	\vspace{0.5cm}
	\vfill

% Bottom of the page
\end{titlepage}
\vspace*{\fill}\begin{center}{\Huge This page intentionally left blank.}\end{center}\vspace*{\fill}\thispagestyle{empty}\clearpage
\pagenumbering{arabic}

\begin{definition}
For any $a > 1$, let $\tau(a)$ denote the smallest integer $n$ such that \[a^n<n!\] 
\end{definition}

\begin{theorem}
$\tau(a)$ is well-defined, i.e., for any $a>1$, there exists an $n$ such that \[a^n<n!\]
\end{theorem}

\begin{proof}
Consider the sequence $h(n)=\frac{a^n}{n!}$. We see that 
\begin{align*}
\lim_{n\to\infty}\frac{h(n+1)}{h(n)}&=\lim_{n\to\infty}\frac{a}{n+1}\\&=0
\end{align*}

By the ratio test, $\displaystyle\sum_{n=0}^{\infty}h(n)$ converges. Thus, the sequence $h(n)=\frac{a^n}{n!}$ converges to 0. Then, for every $\epsilon>0$, there exists an $n_0$ such that, for all $n\geq n_0$, \[\left|\frac{a^n}{n!}\right|<\epsilon\] In particular, taking $\epsilon=1$, we have $a^n<n!$ for all $n\geq n_0$. 
\end{proof}

\begin{lemma}
For all $n>1$, $n!<n^n$.
\end{lemma}

\begin{proof}
Proceed by induction on n. The base case is $2!=2<4=2^2$. Suppose that, for some $k>1$, $k!<k^k$. Then 
\begin{align*}
    (k+1)! &= k!\cdot(k+1)\\
           &< k^k\cdot(k + 1)\text{ by the induction hypothesis}\\
           &< (k+1)^k\cdot(k+1)\\
           &=(k+1)^{k+1}
\end{align*}
\end{proof}

\begin{theorem}
For any integer $n > 1$, $\tau(n) > n$. 
\end{theorem}

\begin{proof}
This follows directly from the above lemma: since $n!<n^n$, $\tau(n)>n$.
\end{proof}

In fact, we can tighten this lower bound to $\tau(n)>2n$. First, we show the following lemma:

\begin{lemma}
For any $n>0$, $n^{2n}\left(4-\frac{2}{n+1}\right)<(n+1)^{2n}$.
\end{lemma}

\begin{proof}
This is equivalent to showing that \[4-\frac{2}{n+1}<\left(\frac{n+1}{n}\right)^{2n}\] However, the right hand side of the above inequality is simply the square of the sequence $\left(1+\frac{1}{n}\right)^n$, which is an increasing sequence that converges to $e$. It suffices, therefore, to find show that the inequality holds for $n=1$, since the left-hand-side is bounded above by 4:

\[4-\frac{2}{1+1}=4-1=3<4=\left(\frac{1+1}{1}\right)^{2\cdot1}\]
\end{proof}

\begin{theorem}
For any integer $n > 2$, $\tau(n)>2n$.
\end{theorem}

\begin{proof}
We shall show by induction that $n^{2n}>(2n)!$. Begin with the base case: \[(2\cdot3)!=6!=720<729=3^6=3^{2\cdot3}\]

Suppose that, for some $k>2$, $(2k)!<k^{2k}$. Then, for $k+1$, we have 
\begin{align*}
(2(k+1))!&=(2k+2)!\\
	 &=(2k)!(2k+1)(2k+2)\\
	 &<k^{2k}(2k+1)(2k+2)\text{ by our Induction Hypothesis}\\
	 &=k^{2k}(k+1)^2\left(4-\frac{2}{k+1}\right)\\
	 &<(k+1)^{2k}(k+1)^2\text{ by Lemma 2}\\
	 &=(k+1)^{2(k+1)}
\end{align*}
Thus, $\tau(n)>2n$. 
\end{proof}


\end{document}
